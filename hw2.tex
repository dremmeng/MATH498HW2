% Options for packages loaded elsewhere
\PassOptionsToPackage{unicode}{hyperref}
\PassOptionsToPackage{hyphens}{url}
%
\documentclass[
]{article}
\usepackage{amsmath,amssymb}
\usepackage{lmodern}
\usepackage{iftex}
\ifPDFTeX
  \usepackage[T1]{fontenc}
  \usepackage[utf8]{inputenc}
  \usepackage{textcomp} % provide euro and other symbols
\else % if luatex or xetex
  \usepackage{unicode-math}
  \defaultfontfeatures{Scale=MatchLowercase}
  \defaultfontfeatures[\rmfamily]{Ligatures=TeX,Scale=1}
\fi
% Use upquote if available, for straight quotes in verbatim environments
\IfFileExists{upquote.sty}{\usepackage{upquote}}{}
\IfFileExists{microtype.sty}{% use microtype if available
  \usepackage[]{microtype}
  \UseMicrotypeSet[protrusion]{basicmath} % disable protrusion for tt fonts
}{}
\makeatletter
\@ifundefined{KOMAClassName}{% if non-KOMA class
  \IfFileExists{parskip.sty}{%
    \usepackage{parskip}
  }{% else
    \setlength{\parindent}{0pt}
    \setlength{\parskip}{6pt plus 2pt minus 1pt}}
}{% if KOMA class
  \KOMAoptions{parskip=half}}
\makeatother
\usepackage{xcolor}
\usepackage[margin=1in]{geometry}
\usepackage{color}
\usepackage{fancyvrb}
\newcommand{\VerbBar}{|}
\newcommand{\VERB}{\Verb[commandchars=\\\{\}]}
\DefineVerbatimEnvironment{Highlighting}{Verbatim}{commandchars=\\\{\}}
% Add ',fontsize=\small' for more characters per line
\usepackage{framed}
\definecolor{shadecolor}{RGB}{248,248,248}
\newenvironment{Shaded}{\begin{snugshade}}{\end{snugshade}}
\newcommand{\AlertTok}[1]{\textcolor[rgb]{0.94,0.16,0.16}{#1}}
\newcommand{\AnnotationTok}[1]{\textcolor[rgb]{0.56,0.35,0.01}{\textbf{\textit{#1}}}}
\newcommand{\AttributeTok}[1]{\textcolor[rgb]{0.77,0.63,0.00}{#1}}
\newcommand{\BaseNTok}[1]{\textcolor[rgb]{0.00,0.00,0.81}{#1}}
\newcommand{\BuiltInTok}[1]{#1}
\newcommand{\CharTok}[1]{\textcolor[rgb]{0.31,0.60,0.02}{#1}}
\newcommand{\CommentTok}[1]{\textcolor[rgb]{0.56,0.35,0.01}{\textit{#1}}}
\newcommand{\CommentVarTok}[1]{\textcolor[rgb]{0.56,0.35,0.01}{\textbf{\textit{#1}}}}
\newcommand{\ConstantTok}[1]{\textcolor[rgb]{0.00,0.00,0.00}{#1}}
\newcommand{\ControlFlowTok}[1]{\textcolor[rgb]{0.13,0.29,0.53}{\textbf{#1}}}
\newcommand{\DataTypeTok}[1]{\textcolor[rgb]{0.13,0.29,0.53}{#1}}
\newcommand{\DecValTok}[1]{\textcolor[rgb]{0.00,0.00,0.81}{#1}}
\newcommand{\DocumentationTok}[1]{\textcolor[rgb]{0.56,0.35,0.01}{\textbf{\textit{#1}}}}
\newcommand{\ErrorTok}[1]{\textcolor[rgb]{0.64,0.00,0.00}{\textbf{#1}}}
\newcommand{\ExtensionTok}[1]{#1}
\newcommand{\FloatTok}[1]{\textcolor[rgb]{0.00,0.00,0.81}{#1}}
\newcommand{\FunctionTok}[1]{\textcolor[rgb]{0.00,0.00,0.00}{#1}}
\newcommand{\ImportTok}[1]{#1}
\newcommand{\InformationTok}[1]{\textcolor[rgb]{0.56,0.35,0.01}{\textbf{\textit{#1}}}}
\newcommand{\KeywordTok}[1]{\textcolor[rgb]{0.13,0.29,0.53}{\textbf{#1}}}
\newcommand{\NormalTok}[1]{#1}
\newcommand{\OperatorTok}[1]{\textcolor[rgb]{0.81,0.36,0.00}{\textbf{#1}}}
\newcommand{\OtherTok}[1]{\textcolor[rgb]{0.56,0.35,0.01}{#1}}
\newcommand{\PreprocessorTok}[1]{\textcolor[rgb]{0.56,0.35,0.01}{\textit{#1}}}
\newcommand{\RegionMarkerTok}[1]{#1}
\newcommand{\SpecialCharTok}[1]{\textcolor[rgb]{0.00,0.00,0.00}{#1}}
\newcommand{\SpecialStringTok}[1]{\textcolor[rgb]{0.31,0.60,0.02}{#1}}
\newcommand{\StringTok}[1]{\textcolor[rgb]{0.31,0.60,0.02}{#1}}
\newcommand{\VariableTok}[1]{\textcolor[rgb]{0.00,0.00,0.00}{#1}}
\newcommand{\VerbatimStringTok}[1]{\textcolor[rgb]{0.31,0.60,0.02}{#1}}
\newcommand{\WarningTok}[1]{\textcolor[rgb]{0.56,0.35,0.01}{\textbf{\textit{#1}}}}
\usepackage{graphicx}
\makeatletter
\def\maxwidth{\ifdim\Gin@nat@width>\linewidth\linewidth\else\Gin@nat@width\fi}
\def\maxheight{\ifdim\Gin@nat@height>\textheight\textheight\else\Gin@nat@height\fi}
\makeatother
% Scale images if necessary, so that they will not overflow the page
% margins by default, and it is still possible to overwrite the defaults
% using explicit options in \includegraphics[width, height, ...]{}
\setkeys{Gin}{width=\maxwidth,height=\maxheight,keepaspectratio}
% Set default figure placement to htbp
\makeatletter
\def\fps@figure{htbp}
\makeatother
\setlength{\emergencystretch}{3em} % prevent overfull lines
\providecommand{\tightlist}{%
  \setlength{\itemsep}{0pt}\setlength{\parskip}{0pt}}
\setcounter{secnumdepth}{-\maxdimen} % remove section numbering
\ifLuaTeX
  \usepackage{selnolig}  % disable illegal ligatures
\fi
\IfFileExists{bookmark.sty}{\usepackage{bookmark}}{\usepackage{hyperref}}
\IfFileExists{xurl.sty}{\usepackage{xurl}}{} % add URL line breaks if available
\urlstyle{same} % disable monospaced font for URLs
\hypersetup{
  pdftitle={MATH 498 HW2},
  pdfauthor={Drew Remmenga},
  hidelinks,
  pdfcreator={LaTeX via pandoc}}

\title{MATH 498 HW2}
\author{Drew Remmenga}
\date{}

\begin{document}
\maketitle

1.a.

\begin{Shaded}
\begin{Highlighting}[]
\FunctionTok{library}\NormalTok{(splines)}
\NormalTok{HW2Test}\OtherTok{\textless{}{-}} \ControlFlowTok{function}\NormalTok{(x)\{}
\FunctionTok{pgamma}\NormalTok{( x, }\AttributeTok{shape =}\DecValTok{10}\NormalTok{, }\AttributeTok{scale =}\NormalTok{.}\DecValTok{02}\NormalTok{) }\SpecialCharTok{{-}}\NormalTok{ .}\DecValTok{5}\SpecialCharTok{*}\NormalTok{x}
\NormalTok{\}}
\NormalTok{xGrid }\OtherTok{\textless{}{-}} \FunctionTok{seq}\NormalTok{( }\DecValTok{0}\NormalTok{,}\DecValTok{1}\NormalTok{, }\AttributeTok{length.out=}\DecValTok{200}\NormalTok{)}
\NormalTok{y}\OtherTok{\textless{}{-}}\FunctionTok{HW2Test}\NormalTok{(xGrid)}
\NormalTok{fit1}\OtherTok{\textless{}{-}}\FunctionTok{lm}\NormalTok{(y }\SpecialCharTok{\textasciitilde{}} \FunctionTok{poly}\NormalTok{(xGrid, }\DecValTok{17}\NormalTok{, }\AttributeTok{raw=}\ConstantTok{TRUE}\NormalTok{))}
\FunctionTok{lm}\NormalTok{(y }\SpecialCharTok{\textasciitilde{}} \FunctionTok{poly}\NormalTok{(xGrid, }\DecValTok{17}\NormalTok{, }\AttributeTok{raw=}\ConstantTok{TRUE}\NormalTok{))}
\end{Highlighting}
\end{Shaded}

\begin{verbatim}
## 
## Call:
## lm(formula = y ~ poly(xGrid, 17, raw = TRUE))
## 
## Coefficients:
##                   (Intercept)   poly(xGrid, 17, raw = TRUE)1  
##                     7.864e-03                     -3.314e+00  
##  poly(xGrid, 17, raw = TRUE)2   poly(xGrid, 17, raw = TRUE)3  
##                     1.851e+02                     -4.771e+03  
##  poly(xGrid, 17, raw = TRUE)4   poly(xGrid, 17, raw = TRUE)5  
##                     6.071e+04                     -4.262e+05  
##  poly(xGrid, 17, raw = TRUE)6   poly(xGrid, 17, raw = TRUE)7  
##                     1.855e+06                     -5.356e+06  
##  poly(xGrid, 17, raw = TRUE)8   poly(xGrid, 17, raw = TRUE)9  
##                     1.060e+07                     -1.447e+07  
## poly(xGrid, 17, raw = TRUE)10  poly(xGrid, 17, raw = TRUE)11  
##                     1.337e+07                     -7.820e+06  
## poly(xGrid, 17, raw = TRUE)12  poly(xGrid, 17, raw = TRUE)13  
##                     2.355e+06                             NA  
## poly(xGrid, 17, raw = TRUE)14  poly(xGrid, 17, raw = TRUE)15  
##                    -1.734e+05                             NA  
## poly(xGrid, 17, raw = TRUE)16  poly(xGrid, 17, raw = TRUE)17  
##                     1.297e+04                             NA
\end{verbatim}

\begin{Shaded}
\begin{Highlighting}[]
\NormalTok{fit2}\OtherTok{\textless{}{-}}\FunctionTok{lm}\NormalTok{(y }\SpecialCharTok{\textasciitilde{}} \FunctionTok{poly}\NormalTok{(xGrid, }\DecValTok{15}\NormalTok{, }\AttributeTok{raw=}\ConstantTok{TRUE}\NormalTok{))}
\FunctionTok{lm}\NormalTok{(y }\SpecialCharTok{\textasciitilde{}} \FunctionTok{poly}\NormalTok{(xGrid, }\DecValTok{15}\NormalTok{, }\AttributeTok{raw=}\ConstantTok{TRUE}\NormalTok{))}
\end{Highlighting}
\end{Shaded}

\begin{verbatim}
## 
## Call:
## lm(formula = y ~ poly(xGrid, 15, raw = TRUE))
## 
## Coefficients:
##                   (Intercept)   poly(xGrid, 15, raw = TRUE)1  
##                     4.063e-03                     -2.274e+00  
##  poly(xGrid, 15, raw = TRUE)2   poly(xGrid, 15, raw = TRUE)3  
##                     1.279e+02                     -3.457e+03  
##  poly(xGrid, 15, raw = TRUE)4   poly(xGrid, 15, raw = TRUE)5  
##                     4.458e+04                     -3.062e+05  
##  poly(xGrid, 15, raw = TRUE)6   poly(xGrid, 15, raw = TRUE)7  
##                     1.274e+06                     -3.451e+06  
##  poly(xGrid, 15, raw = TRUE)8   poly(xGrid, 15, raw = TRUE)9  
##                     6.292e+06                     -7.762e+06  
## poly(xGrid, 15, raw = TRUE)10  poly(xGrid, 15, raw = TRUE)11  
##                     6.324e+06                     -3.162e+06  
## poly(xGrid, 15, raw = TRUE)12  poly(xGrid, 15, raw = TRUE)13  
##                     7.798e+05                             NA  
## poly(xGrid, 15, raw = TRUE)14  poly(xGrid, 15, raw = TRUE)15  
##                    -2.989e+04                             NA
\end{verbatim}

\begin{Shaded}
\begin{Highlighting}[]
\NormalTok{fit3}\OtherTok{\textless{}{-}}\FunctionTok{lm}\NormalTok{(y }\SpecialCharTok{\textasciitilde{}} \FunctionTok{poly}\NormalTok{(xGrid, }\DecValTok{13}\NormalTok{, }\AttributeTok{raw=}\ConstantTok{TRUE}\NormalTok{))}
\FunctionTok{lm}\NormalTok{(y }\SpecialCharTok{\textasciitilde{}} \FunctionTok{poly}\NormalTok{(xGrid, }\DecValTok{13}\NormalTok{, }\AttributeTok{raw=}\ConstantTok{TRUE}\NormalTok{))}
\end{Highlighting}
\end{Shaded}

\begin{verbatim}
## 
## Call:
## lm(formula = y ~ poly(xGrid, 13, raw = TRUE))
## 
## Coefficients:
##                   (Intercept)   poly(xGrid, 13, raw = TRUE)1  
##                    -8.951e-03                      7.375e-01  
##  poly(xGrid, 13, raw = TRUE)2   poly(xGrid, 13, raw = TRUE)3  
##                    -1.645e+01                     -5.663e+02  
##  poly(xGrid, 13, raw = TRUE)4   poly(xGrid, 13, raw = TRUE)5  
##                     1.361e+04                     -1.058e+05  
##  poly(xGrid, 13, raw = TRUE)6   poly(xGrid, 13, raw = TRUE)7  
##                     4.349e+05                     -1.091e+06  
##  poly(xGrid, 13, raw = TRUE)8   poly(xGrid, 13, raw = TRUE)9  
##                     1.759e+06                     -1.840e+06  
## poly(xGrid, 13, raw = TRUE)10  poly(xGrid, 13, raw = TRUE)11  
##                     1.208e+06                     -4.536e+05  
## poly(xGrid, 13, raw = TRUE)12  poly(xGrid, 13, raw = TRUE)13  
##                     7.435e+04                             NA
\end{verbatim}

\begin{Shaded}
\begin{Highlighting}[]
\NormalTok{fit4}\OtherTok{\textless{}{-}}\FunctionTok{lm}\NormalTok{(y }\SpecialCharTok{\textasciitilde{}} \FunctionTok{poly}\NormalTok{(xGrid, }\DecValTok{11}\NormalTok{, }\AttributeTok{raw=}\ConstantTok{TRUE}\NormalTok{))}
\FunctionTok{lm}\NormalTok{(y }\SpecialCharTok{\textasciitilde{}} \FunctionTok{poly}\NormalTok{(xGrid, }\DecValTok{11}\NormalTok{, }\AttributeTok{raw=}\ConstantTok{TRUE}\NormalTok{))}
\end{Highlighting}
\end{Shaded}

\begin{verbatim}
## 
## Call:
## lm(formula = y ~ poly(xGrid, 11, raw = TRUE))
## 
## Coefficients:
##                   (Intercept)   poly(xGrid, 11, raw = TRUE)1  
##                    -2.856e-02                      4.507e+00  
##  poly(xGrid, 11, raw = TRUE)2   poly(xGrid, 11, raw = TRUE)3  
##                    -1.706e+02                      2.076e+03  
##  poly(xGrid, 11, raw = TRUE)4   poly(xGrid, 11, raw = TRUE)5  
##                    -1.054e+04                      2.688e+04  
##  poly(xGrid, 11, raw = TRUE)6   poly(xGrid, 11, raw = TRUE)7  
##                    -3.220e+04                      3.614e+02  
##  poly(xGrid, 11, raw = TRUE)8   poly(xGrid, 11, raw = TRUE)9  
##                     4.987e+04                     -6.389e+04  
## poly(xGrid, 11, raw = TRUE)10  poly(xGrid, 11, raw = TRUE)11  
##                     3.509e+04                     -7.484e+03
\end{verbatim}

1.b.

\begin{Shaded}
\begin{Highlighting}[]
\FunctionTok{plot}\NormalTok{(xGrid, }\FunctionTok{HW2Test}\NormalTok{(xGrid), }\AttributeTok{col=}\StringTok{"Red"}\NormalTok{)}
\NormalTok{pred1}\OtherTok{\textless{}{-}} \FunctionTok{predict}\NormalTok{(fit1)}
\NormalTok{pred2}\OtherTok{\textless{}{-}} \FunctionTok{predict}\NormalTok{(fit2)}
\NormalTok{pred3}\OtherTok{\textless{}{-}} \FunctionTok{predict}\NormalTok{(fit3)}
\NormalTok{pred4}\OtherTok{\textless{}{-}} \FunctionTok{predict}\NormalTok{(fit4)}
\FunctionTok{lines}\NormalTok{(xGrid, pred1, }\AttributeTok{col=}\StringTok{\textquotesingle{}Blue\textquotesingle{}}\NormalTok{, }\AttributeTok{lwd=}\DecValTok{2}\NormalTok{)}
\FunctionTok{lines}\NormalTok{(xGrid, pred2, }\AttributeTok{col=}\StringTok{\textquotesingle{}Green\textquotesingle{}}\NormalTok{, }\AttributeTok{lwd=}\DecValTok{2}\NormalTok{)}
\FunctionTok{lines}\NormalTok{(xGrid, pred3, }\AttributeTok{col=}\StringTok{\textquotesingle{}Black\textquotesingle{}}\NormalTok{, }\AttributeTok{lwd=}\DecValTok{2}\NormalTok{)}
\FunctionTok{lines}\NormalTok{(xGrid, pred4, }\AttributeTok{col=}\StringTok{\textquotesingle{}Yellow\textquotesingle{}}\NormalTok{, }\AttributeTok{lwd=}\DecValTok{2}\NormalTok{)}
\FunctionTok{legend}\NormalTok{(}\StringTok{"topleft"}\NormalTok{, }\AttributeTok{legend=}\FunctionTok{c}\NormalTok{(}\StringTok{"Actual"}\NormalTok{, }\StringTok{"Degree 17"}\NormalTok{, }\StringTok{"Degree 15"}\NormalTok{, }\StringTok{"Degree 13"}\NormalTok{, }\StringTok{"Degree 11"}\NormalTok{),}
       \AttributeTok{col=}\FunctionTok{c}\NormalTok{(}\StringTok{"Red"}\NormalTok{, }\StringTok{"Blue"}\NormalTok{, }\StringTok{"Green"}\NormalTok{, }\StringTok{"Black"}\NormalTok{, }\StringTok{"Yellow"}\NormalTok{), }\AttributeTok{lty=}\DecValTok{1}\SpecialCharTok{:}\DecValTok{2}\NormalTok{, }\AttributeTok{cex=}\FloatTok{0.8}\NormalTok{)}
\FunctionTok{title}\NormalTok{(}\StringTok{"Different degree Polynomials fit to the Gamma function"}\NormalTok{)}
\end{Highlighting}
\end{Shaded}

\includegraphics{hw2_files/figure-latex/unnamed-chunk-2-1.pdf} 1.c.
Degrees 17,15,13, and 11.

\begin{Shaded}
\begin{Highlighting}[]
\FunctionTok{sqrt}\NormalTok{( }\FunctionTok{mean}\NormalTok{( ( y }\SpecialCharTok{{-}}\NormalTok{ pred1)}\SpecialCharTok{\^{}}\DecValTok{2}\NormalTok{ ))}
\end{Highlighting}
\end{Shaded}

\begin{verbatim}
## [1] 0.001742046
\end{verbatim}

\begin{Shaded}
\begin{Highlighting}[]
\FunctionTok{sqrt}\NormalTok{( }\FunctionTok{mean}\NormalTok{( ( y }\SpecialCharTok{{-}}\NormalTok{ pred2)}\SpecialCharTok{\^{}}\DecValTok{2}\NormalTok{ ))}
\end{Highlighting}
\end{Shaded}

\begin{verbatim}
## [1] 0.002157356
\end{verbatim}

\begin{Shaded}
\begin{Highlighting}[]
\FunctionTok{sqrt}\NormalTok{( }\FunctionTok{mean}\NormalTok{( ( y }\SpecialCharTok{{-}}\NormalTok{ pred3)}\SpecialCharTok{\^{}}\DecValTok{2}\NormalTok{ ))}
\end{Highlighting}
\end{Shaded}

\begin{verbatim}
## [1] 0.004626954
\end{verbatim}

\begin{Shaded}
\begin{Highlighting}[]
\FunctionTok{sqrt}\NormalTok{( }\FunctionTok{mean}\NormalTok{( ( y }\SpecialCharTok{{-}}\NormalTok{ pred4)}\SpecialCharTok{\^{}}\DecValTok{2}\NormalTok{ ))}
\end{Highlighting}
\end{Shaded}

\begin{verbatim}
## [1] 0.007413881
\end{verbatim}

2.a.

\begin{Shaded}
\begin{Highlighting}[]
\NormalTok{xGrid}\OtherTok{\textless{}{-}} \FunctionTok{seq}\NormalTok{( }\DecValTok{0}\NormalTok{,}\DecValTok{1}\NormalTok{,}\AttributeTok{length.out=}\DecValTok{150}\NormalTok{)}
\NormalTok{KN}\OtherTok{\textless{}{-}} \FunctionTok{seq}\NormalTok{( }\DecValTok{0}\NormalTok{,}\DecValTok{1}\NormalTok{,}\AttributeTok{length.out=}\DecValTok{10}\NormalTok{)}
\NormalTok{naturalSplineBasis }\OtherTok{\textless{}{-}} \ControlFlowTok{function}\NormalTok{(sGrid,}
\NormalTok{                               sKnots,}
                               \AttributeTok{degree =} \DecValTok{3}\NormalTok{,}
                               \AttributeTok{derivative =} \DecValTok{0}\NormalTok{) \{}
\NormalTok{  boundaryKnots}\OtherTok{\textless{}{-}} \FunctionTok{c}\NormalTok{( }\FunctionTok{min}\NormalTok{(sKnots),}\FunctionTok{max}\NormalTok{(sKnots))}
\NormalTok{  sKnots0}\OtherTok{\textless{}{-}} \FunctionTok{c}\NormalTok{( }\FunctionTok{rep}\NormalTok{( boundaryKnots[}\DecValTok{1}\NormalTok{],degree),}\FunctionTok{sort}\NormalTok{(sKnots),}
               \FunctionTok{rep}\NormalTok{( boundaryKnots[}\DecValTok{2}\NormalTok{],degree) )}
\NormalTok{  testRight}\OtherTok{\textless{}{-}}\NormalTok{ sGrid }\SpecialCharTok{\textless{}} \FunctionTok{min}\NormalTok{(sKnots) }
\NormalTok{  testLeft }\OtherTok{\textless{}{-}}\NormalTok{ sGrid }\SpecialCharTok{\textgreater{}} \FunctionTok{max}\NormalTok{(sKnots)             }
  \ControlFlowTok{if}\NormalTok{( }\FunctionTok{any}\NormalTok{(testRight }\SpecialCharTok{|}\NormalTok{testLeft) )}
\NormalTok{  \{}\FunctionTok{stop}\NormalTok{(}\StringTok{"some points for evaluation outside knot range."}\NormalTok{)\}}
               
\NormalTok{  basis }\OtherTok{\textless{}{-}} \FunctionTok{splineDesign}\NormalTok{(sKnots0, sGrid,}
                        \AttributeTok{ord=}\NormalTok{ degree}\SpecialCharTok{+}\DecValTok{1}\NormalTok{, }\AttributeTok{outer.ok=}\ConstantTok{TRUE}\NormalTok{,}
                        \AttributeTok{derivs=}\NormalTok{derivative)}
  \CommentTok{\# set up constraints to enforce natural BCs.}
\NormalTok{  const }\OtherTok{\textless{}{-}} \FunctionTok{splineDesign}\NormalTok{(sKnots0, boundaryKnots, }\AttributeTok{ord =}\NormalTok{ degree}\SpecialCharTok{+}\DecValTok{1}\NormalTok{,}
                        \AttributeTok{derivs =} \FunctionTok{c}\NormalTok{(}\DecValTok{2}\NormalTok{,}\DecValTok{2}\NormalTok{)) }
\NormalTok{  qr.const }\OtherTok{\textless{}{-}} \FunctionTok{qr}\NormalTok{(}\FunctionTok{t}\NormalTok{(const))}
\NormalTok{  QBasis}\OtherTok{\textless{}{-}} \FunctionTok{t}\NormalTok{(}\FunctionTok{qr.qty}\NormalTok{( qr.const, }\FunctionTok{t}\NormalTok{(basis) ))}
\NormalTok{  basis }\OtherTok{\textless{}{-}}\NormalTok{ QBasis[,}\SpecialCharTok{{-}}\NormalTok{(}\DecValTok{1}\SpecialCharTok{:}\DecValTok{2}\NormalTok{)]}
\NormalTok{  basis}
  
  \FunctionTok{return}\NormalTok{( basis )}
  
\NormalTok{\}}
\NormalTok{B}\OtherTok{\textless{}{-}} \FunctionTok{naturalSplineBasis}\NormalTok{( xGrid, }\AttributeTok{sKnots=}\NormalTok{KN)}
\FunctionTok{matplot}\NormalTok{( xGrid, B, }\AttributeTok{type=}\StringTok{"l"}\NormalTok{, }\AttributeTok{lty=}\DecValTok{1}\NormalTok{)}
\end{Highlighting}
\end{Shaded}

\includegraphics{hw2_files/figure-latex/unnamed-chunk-4-1.pdf} Just one.
2.b.

\begin{Shaded}
\begin{Highlighting}[]
\NormalTok{NBSFit}\OtherTok{\textless{}{-}} \ControlFlowTok{function}\NormalTok{( x, y, xGrid)\{}
\NormalTok{N}\OtherTok{\textless{}{-}} \FunctionTok{length}\NormalTok{( x)}
\NormalTok{AData}\OtherTok{\textless{}{-}} \FunctionTok{naturalSplineBasis}\NormalTok{( x, }\AttributeTok{sKnots=}\NormalTok{x)}
\NormalTok{coef}\OtherTok{\textless{}{-}} \FunctionTok{solve}\NormalTok{( AData, y)}
\NormalTok{AGrid}\OtherTok{\textless{}{-}} \FunctionTok{naturalSplineBasis}\NormalTok{( xGrid, }\AttributeTok{sKnots=}\NormalTok{x )}
\NormalTok{yFit }\OtherTok{\textless{}{-}}\NormalTok{AGrid}\SpecialCharTok{\%*\%}\NormalTok{coef}
\FunctionTok{return}\NormalTok{( yFit)}
\NormalTok{\}}
\NormalTok{y }\OtherTok{\textless{}{-}} \FunctionTok{HW2Test}\NormalTok{(xGrid)}
\FunctionTok{NBSFit}\NormalTok{(xGrid,y,xGrid)}
\end{Highlighting}
\end{Shaded}

\begin{verbatim}
##                 [,1]
##   [1,] -2.220446e-16
##   [2,] -3.355705e-03
##   [3,] -6.711407e-03
##   [4,] -1.006700e-02
##   [5,] -1.342126e-02
##   [6,] -1.676781e-02
##   [7,] -2.008511e-02
##   [8,] -2.331980e-02
##   [9,] -2.636568e-02
##  [10,] -2.904338e-02
##  [11,] -3.108679e-02
##  [12,] -3.214119e-02
##  [13,] -3.177501e-02
##  [14,] -2.950489e-02
##  [15,] -2.483043e-02
##  [16,] -1.727415e-02
##  [17,] -6.421174e-03
##  [18,]  8.046000e-03
##  [19,]  2.632126e-02
##  [20,]  4.845925e-02
##  [21,]  7.437099e-02
##  [22,]  1.038300e-01
##  [23,]  1.364874e-01
##  [24,]  1.718933e-01
##  [25,]  2.095226e-01
##  [26,]  2.488027e-01
##  [27,]  2.891401e-01
##  [28,]  3.299456e-01
##  [29,]  3.706559e-01
##  [30,]  4.107509e-01
##  [31,]  4.497671e-01
##  [32,]  4.873061e-01
##  [33,]  5.230393e-01
##  [34,]  5.567090e-01
##  [35,]  5.881264e-01
##  [36,]  6.171677e-01
##  [37,]  6.437677e-01
##  [38,]  6.679132e-01
##  [39,]  6.896354e-01
##  [40,]  7.090017e-01
##  [41,]  7.261088e-01
##  [42,]  7.410753e-01
##  [43,]  7.540354e-01
##  [44,]  7.651329e-01
##  [45,]  7.745168e-01
##  [46,]  7.823371e-01
##  [47,]  7.887410e-01
##  [48,]  7.938712e-01
##  [49,]  7.978632e-01
##  [50,]  8.008444e-01
##  [51,]  8.029328e-01
##  [52,]  8.042372e-01
##  [53,]  8.048563e-01
##  [54,]  8.048795e-01
##  [55,]  8.043867e-01
##  [56,]  8.034492e-01
##  [57,]  8.021297e-01
##  [58,]  8.004837e-01
##  [59,]  7.985594e-01
##  [60,]  7.963989e-01
##  [61,]  7.940386e-01
##  [62,]  7.915097e-01
##  [63,]  7.888391e-01
##  [64,]  7.860498e-01
##  [65,]  7.831612e-01
##  [66,]  7.801898e-01
##  [67,]  7.771498e-01
##  [68,]  7.740527e-01
##  [69,]  7.709085e-01
##  [70,]  7.677254e-01
##  [71,]  7.645103e-01
##  [72,]  7.612689e-01
##  [73,]  7.580060e-01
##  [74,]  7.547254e-01
##  [75,]  7.514305e-01
##  [76,]  7.481239e-01
##  [77,]  7.448078e-01
##  [78,]  7.414839e-01
##  [79,]  7.381538e-01
##  [80,]  7.348186e-01
##  [81,]  7.314794e-01
##  [82,]  7.281368e-01
##  [83,]  7.247916e-01
##  [84,]  7.214443e-01
##  [85,]  7.180953e-01
##  [86,]  7.147449e-01
##  [87,]  7.113934e-01
##  [88,]  7.080411e-01
##  [89,]  7.046880e-01
##  [90,]  7.013344e-01
##  [91,]  6.979804e-01
##  [92,]  6.946260e-01
##  [93,]  6.912714e-01
##  [94,]  6.879165e-01
##  [95,]  6.845614e-01
##  [96,]  6.812062e-01
##  [97,]  6.778509e-01
##  [98,]  6.744955e-01
##  [99,]  6.711401e-01
## [100,]  6.677845e-01
## [101,]  6.644290e-01
## [102,]  6.610734e-01
## [103,]  6.577178e-01
## [104,]  6.543622e-01
## [105,]  6.510065e-01
## [106,]  6.476509e-01
## [107,]  6.442952e-01
## [108,]  6.409395e-01
## [109,]  6.375838e-01
## [110,]  6.342281e-01
## [111,]  6.308724e-01
## [112,]  6.275167e-01
## [113,]  6.241610e-01
## [114,]  6.208053e-01
## [115,]  6.174496e-01
## [116,]  6.140939e-01
## [117,]  6.107382e-01
## [118,]  6.073825e-01
## [119,]  6.040268e-01
## [120,]  6.006711e-01
## [121,]  5.973154e-01
## [122,]  5.939597e-01
## [123,]  5.906040e-01
## [124,]  5.872483e-01
## [125,]  5.838926e-01
## [126,]  5.805369e-01
## [127,]  5.771812e-01
## [128,]  5.738255e-01
## [129,]  5.704698e-01
## [130,]  5.671141e-01
## [131,]  5.637584e-01
## [132,]  5.604027e-01
## [133,]  5.570470e-01
## [134,]  5.536913e-01
## [135,]  5.503356e-01
## [136,]  5.469799e-01
## [137,]  5.436242e-01
## [138,]  5.402685e-01
## [139,]  5.369128e-01
## [140,]  5.335570e-01
## [141,]  5.302013e-01
## [142,]  5.268456e-01
## [143,]  5.234899e-01
## [144,]  5.201342e-01
## [145,]  5.167785e-01
## [146,]  5.134228e-01
## [147,]  5.100671e-01
## [148,]  5.067114e-01
## [149,]  5.033557e-01
## [150,]  5.000000e-01
\end{verbatim}

Sines and cosines are mutually orthogonal.
\(\int_{-\infty}^{\infty} \sqrt(2) cos(2\pi x)*\sqrt(2) cos(2\pi x)=1\)
\(\int_{-\infty}^{\infty} \sqrt(2) sin(2\pi x)*\sqrt(2) sin(2\pi x)=1\)
\(\int_{-\infty}^{\infty} \sqrt(2) cos(4\pi x)*\sqrt(2) cos(4\pi x)=1\)
\(\int_{-\infty}^{\infty} \sqrt(2) sin(4\pi x)*\sqrt(2) sin(4\pi x)=1\)

\end{document}
